% !TeX spellcheck = it_IT
\documentclass{llncs}
%%%%%%%%%%%%%%%%%%%%%%%%%%%%%%%%%%%%%%%%%%%%%%%%%%%%%%%%%%%
%% package sillabazione italiana e uso lettere accentate
\usepackage[latin1]{inputenc}
\usepackage[english]{babel}
\usepackage[T1]{fontenc}
%%%%%%%%%%%%%%%%%%%%%%%%%%%%%%%%%%%%%%%%%%%%%%%%%%%%%%%%%%%%%

\usepackage{url}
\usepackage{xspace}
\usepackage{amsmath}
\usepackage{pdfpages}


\makeatletter
%%%%%%%%%%%%%%%%%%%%%%%%%%%%%% User specified LaTeX commands.
\usepackage{manifest}
\usepackage{listings}
\usepackage{textcomp}
\makeatother

\usepackage{tikz}
\usetikzlibrary{arrows,automata}

\newcommand{\java}{\textsf{Java}}
\newcommand{\contact}{\emph{Contact}}
\newcommand{\corecl}{\texttt{corecl}}
\newcommand{\medcl}{\texttt{medcl}}
\newcommand{\msgcl}{\texttt{msgcl}}
\newcommand{\android}{\texttt{Android}}
\newcommand{\dsl}{\texttt{DSL}}
\newcommand{\jazz}{\texttt{Jazz}}
\newcommand{\rtc}{\texttt{RTC}}
\newcommand{\ide}{\texttt{Contact-ide}}
\newcommand{\xtext}{\texttt{XText}}
\newcommand{\xpand}{\texttt{Xpand}}
\newcommand{\xtend}{\texttt{Xtend}}
\newcommand{\pojo}{\texttt{POJO}}
\newcommand{\junit}{\texttt{JUnit}}

\newcommand{\action}[1]{\texttt{#1}\xspace}
\newcommand{\code}[1]{{\small{\texttt{#1}}}\xspace}
\newcommand{\codescript}[1]{{\scriptsize{\texttt{#1}}}\xspace}

% Cross-referencing
\newcommand{\labelsec}[1]{\label{sec:#1}}
\newcommand{\xs}[1]{\sectionname~\ref{sec:#1}}
\newcommand{\xsp}[1]{\sectionname~\ref{sec:#1} \onpagename~\pageref{sec:#1}}
\newcommand{\labelssec}[1]{\label{ssec:#1}}
\newcommand{\xss}[1]{\subsectionname~\ref{ssec:#1}}
\newcommand{\xssp}[1]{\subsectionname~\ref{ssec:#1} \onpagename~\pageref{ssec:#1}}
\newcommand{\labelsssec}[1]{\label{sssec:#1}}
\newcommand{\xsss}[1]{\subsectionname~\ref{sssec:#1}}
\newcommand{\xsssp}[1]{\subsectionname~\ref{sssec:#1} \onpagename~\pageref{sssec:#1}}
\newcommand{\labelfig}[1]{\label{fig:#1}}
\newcommand{\xf}[1]{\figurename~\ref{fig:#1}}
\newcommand{\xfp}[1]{\figurename~\ref{fig:#1} \onpagename~\pageref{fig:#1}}
\newcommand{\labeltab}[1]{\label{tab:#1}}
\newcommand{\xt}[1]{\tablename~\ref{tab:#1}}
\newcommand{\xtp}[1]{\tablename~\ref{tab:#1} \onpagename~\pageref{tab:#1}}
% Category Names
\newcommand{\sectionname}{Section}
\newcommand{\subsectionname}{Subsection}
\newcommand{\sectionsname}{Sections}
\newcommand{\subsectionsname}{Subsections}
\newcommand{\secname}{\sectionname}
\newcommand{\ssecname}{\subsectionname}
\newcommand{\secsname}{\sectionsname}
\newcommand{\ssecsname}{\subsectionsname}
\newcommand{\onpagename}{on page}

\newcommand{\student}{Gruppo 3 Aimi Niccol\'o, Gallegati Mattia, Murgia Antonio, Zanotti Andrea }
\newcommand{\studentEmail}{niccolo.aimi@studio.unibo.it; mattia.gallegati2@studio.unibo.it; antonio.murgia2@studio.unibo.it; andrea.zanotti9@studio.unibo.it}
\newcommand{\xfaculty}{II Faculty of Engineering}
\newcommand{\xunibo}{Alma Mater Studiorum -- University of Bologna}
\newcommand{\xaddrBO}{viale Risorgimento 2}
\newcommand{\xaddrCE}{via Venezia 52}
\newcommand{\xcityBO}{40136 Bologna, Italy}
\newcommand{\xcityCE}{47023 Cesena, Italy}

%
% Comments
%
%%% \newcommand{\todo}[1]{\bf{TODO:}\emph{#1}}


\begin{document}

\title{Final Theme System\\
 process report}

%%% \author{\xauthA \and \xauthB}
\author{\student}

\institute{%
%%%  \xunibo\\\xaddrCE, \xcityCE\\\email{\{nameA.studentA, nameB.studentB\}@studio.unibo.it}
  \xunibo\\\xaddrBO, \xcityBO\\\email\ {\studentEmail}
}

\maketitle

%% \begin{abstract}
%% \footnotesize
%%This a Latex template to be used for the reports of Software Engineering.
%%\keywords{Software engineering, managed software development, reports, ....}
%%\end{abstract}

%%% \sloppy

%===========================================================================
\section{Introduction}
\labelsec{intro}
%===========================================================================

%===========================================================================
\section{Vision}
\labelsec{Vision}
Here we're collecting some visions that will inspire the development of this project and of software in general:\\
\begin{itemize}
	\item There's no code without project, there's no project without problem analysis and there's no problem without requirements;
	\item There's no code without tests. That means that tests must be developed before the development of the software product;
	\item There's no code without documentation;
	\item The team that develops the tests should be different from the one who will realize the project;
	\item We should use a top down approach during the development phase and bottom-up approach during the implementation phase;
	\item We should always check the existence of prior projects, trends or patterns regarding the technological domain we are facing. If does exist something we should study it, not only to take advantage of it during the development but also to recognize its limits and to purpose some innovative approach coming from different technological domains;
	\item The development of the project should be technology agnostic;
	\item We should find or create a formal language able to describe the results of the problem analysis. It should be similar to spoken language. This language could take advantage of different programming paradigms (functional, declarative and so on) and different programming models (actor model, message passing model and so on). This language should not be tied to any technology implementation. There should be one or more parsers/compilers for this language that will generate source code for a specific platform. This generated code should be used as the skeleton of the real product.
\end{itemize}
%===========================================================================

%===========================================================================
\section{Goals}
The main goal is discuss how some change (monotonic extensions) of the requirements impact on a product whose production is based on 'formal' and 'technology independent' artifacts rather than on ad-hoc code.
\labelsec{Goals}
%===========================================================================

%===========================================================================
\section{Requirements}
\labelsec{Requirements}
Design and build a (prototype of a) software system that, with reference to a differential drive robot (called from now on robot):
\begin{itemize}
	\item allows a user to select between a 'learning phase' and a 'autonomous phase'
    during the learning phase, the user can send a sequence of move commands (e.g. forward, backward, left right, stop) to the robot. The robot must not only execute each command but it must also record the whole sequence of commands until the user decides to terminate the learning phase;
    \item after the termination of the learning phase, the user can tell the robot to enter the autonomous phase in a 'direct' or in a 'reverse' mode. During this phase the robot executes in autonomous way the sequence of moves it has learned, by complementing each move (e.g. forward->backward) if the selected mode is reverse;
    \item during the autonomous phase, the robot must be able to execute (as soon as possible) a stop command sent by the user.
\end{itemize}
After the development of this prototype, consider the possibility to enhance the functional capabilities of the robot, by allowing it:
\begin{itemize}
	\item to perceive an obstacle during the autonomous phase and, once the obstacle is detected, to execute some alternative behaviour (in term of moves).
\end{itemize}

%===========================================================================

 
%===========================================================================
\section{Requirement analysis}
\labelsec{ReqAnalysis}
%===========================================================================
\subsection{Use cases}
\labelssec{UseCases}

\subsection{Scenarios}
\labelssec{Scenarios}

\subsection{(Domain)model}
\labelssec{(Domain)model}
\paragraph{Structure}
\subsubsection{ROBOT}
We consider the robot as a reactive and atomic entity.
\subsubsection{REMOTE}
We consider the remote as a proactive and atomic entity.
\paragraph{Interaction}
The interaction between the entities of the system is described with this sequence diagram:
\begin{center}
   	\includegraphics[width=13cm]{img/interactionRobot.png}\\
\end{center}
\paragraph{Behaviour}
\subsubsection{ROBOT}
The behaviour of the robot is described with this FSM:\\
\begin{center}
   	\includegraphics[width=12.5cm]{img/fsmRobot.png}\\
\end{center}
\subsubsection{REMOTE}
The behaviour of the remote is described with this FSM:\\
\begin{center}
   	\includegraphics[width=2cm]{img/fsmRemoteRequirementAnalysis.png}\\
\end{center}
\subsection{Test plan}
\labelsec{Test plan}
%===========================================================================
\section{Problem analysis}
\labelsec{ProblemAnalysis}
Per risolvere il problema di scalabilit\`a e mantenere le entit\`a disaccoppiate faremo utilizzo del pattern \texttt{Observer}: il Button espone un'interfaccia tramite la quale le entit\`a possono registrarsi.\\
Dividiamo i componenti dell'architettura in Attivi e Passivi.\\
I componenti Passivi generalmente generano eventi, i componenti Attivi possono ricevere comandi. Per la realizzazione di tale funzionalit\`a verr\`a utilizzato il pattern \texttt{Command}.\\
%\`E inoltre necessario accordarsi su un linguaggio di interazione, il quale necessita di uno standard di comunicazione. Lo standard di comunicazione sar\`a indispensabile per Integration Test.
%===========================================================================
\subsection{Logic architecture}
L'architettura logica \`e esplicitata in figura.
\begin{center}
\includegraphics{img/graphs/Analisi_BLS.pdf}
\end{center}
\begin{center}

\end{center}

\subsection{Abstraction gap}

\subsection{Risk analysis}
Nell'interfaccia del Button, oltre al metodo accessor \texttt{isPressed} deve essere presente anche il metodo \texttt{addObserver}, ci\`o \`e risultato necessario dai colloqui con il committente (durante l'analisi dei requisiti).
%===========================================================================
\section{Work plan}
\labelsec{wplan}
%===========================================================================

%===========================================================================
\section{Project}
\labelsec{Project}
L'architettura progettuale \`e esplicitata in figura.\\
\url{https://github.com/iBelliDiISS/ButtonLedJava}
\begin{center}
\includegraphics[scale=0.6]{img/graphs/Progettazione_BLS.pdf}
\end{center}
%===========================================================================

\subsection{Structure}
\subsection{Interaction}
\begin{center}
\includegraphics{img/graphs/Interaction.pdf}
\end{center}
\subsection{Behavior}
%===========================================================================
\section{Implementation}
\labelsec{Implementation}
%===========================================================================

%===========================================================================
\section{Testing}
\labelsec{testing}
%===========================================================================

%===========================================================================
\section{Deployment}
\labelsec{Deployment}
%===========================================================================

%===========================================================================
\section{Maintenance}
\labelsec{Maintenance}
%===========================================================================

%===========================================================================
\section{Discussion}
\paragraph{Requirement analysis}

\labelsec{Discussion}
%===========================================================================
_discutere dell'evento nel caso sia generato dal nulla o da altro
_dire come rappresentarlo visto che non possiamo farlo in uml
_introduzione del sistema ad attori rispetto a java
_problema se in analisi dei requisiti parlare dell'indipendenza dalla tecnologia
_cambiamento dell'fsm del remote in fase successiva
%===========================================================================
\newpage
\appendix
\section{Codice}
\section{Information about the author}
\labelsec{Author}
%===========================================================================
\appendix
%===========================================================================
\vskip.5cm
%%% \begin{figure}
\begin{tabular}{ | c |  }
	\hline
	Photo of the Author\\
   	\includegraphics[width=5cm]{img/zano.jpg}\\
	\hline
\end{tabular}
 
\appendix


\bibliographystyle{abbrv}
\bibliography{biblio}

\end{document}












